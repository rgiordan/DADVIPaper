%%%%%%%%%%%%%%%%%%%%%%%%%%%%%%%%%%%%%%
%%%%%%%%%%%%%%%%%%%%%%%%%%%%%%%%%%%%%%
% Do not edit the TeX file your work
% will be overwritten.  Edit the RnW
% file instead.
%%%%%%%%%%%%%%%%%%%%%%%%%%%%%%%%%%%%%%
%%%%%%%%%%%%%%%%%%%%%%%%%%%%%%%%%%%%%%





\newcommand{\ARMNumModels}{53}
\newcommand{\MCParamDim}{124}
\newcommand{\OccParamDim}{1,884}
\newcommand{\PotusParamDim}{15,098}
\newcommand{\TennisParamDim}{5,014}
\newcommand{\ARMMinParamDim}{2}
\newcommand{\ARMMedParamDim}{5}
\newcommand{\ARMMaxParamDim}{176}
\newcommand{\TennisNUTSMinutes}{57}
\newcommand{\PotusNUTSMinutes}{643}
\newcommand{\MCNUTSMinutes}{597}
\newcommand{\OccNUTSMinutes}{251}
\newcommand{\ARMMinNUTSSeconds}{15}
\newcommand{\ARMMedNUTSSeconds}{39}
\newcommand{\ARMMaxNUTSMinutes}{16}
\newcommand{\DADVINumDraws}{30}
\newcommand{\CoverageNumBins}{100}
\newcommand{\CoverageNumBinsPotus}{100}
\newcommand{\ArmModels}{separation (2), wells\_dist100 (2), nes2000\_vote (2), wells\_d100ars (3), earn\_height (3), sesame\_one\_pred\_b (3), radon\_complete\_pool (3), earnings1 (3), kidscore\_momiq (3), kidscore\_momhs (3), electric\_one\_pred (3), sesame\_one\_pred\_a (3), sesame\_one\_pred\_2b (3), logearn\_height (3), electric\_multi\_preds (4), congress (4), wells\_interaction\_c (4), earnings2 (4), logearn\_logheight (4), kidiq\_multi\_preds (4), wells\_dae (4), wells\_interaction (4), logearn\_height\_male (4), ideo\_reparam (5), logearn\_interaction (5), kidscore\_momwork (5), kidiq\_interaction (5), wells\_dae\_c (5), mesquite\_volume (5), earnings\_interactions (5), wells\_dae\_inter (5), wells\_daae\_c (6), wells\_dae\_inter\_c (7), mesquite\_vash (7), wells\_predicted\_log (7), mesquite (8), mesquite\_vas (8), mesquite\_log (8), sesame\_multi\_preds\_3b (9), sesame\_multi\_preds\_3a (9), pilots (17), election88 (53), radon\_intercept (88), radon\_no\_pool (89), radon\_group (90), electric (100), electric\_1b (101), electric\_1a (109), electric\_1c (114), hiv (170), hiv\_inter (171), radon\_vary\_si (174), radon\_inter\_vary (176)}
\newcommand{\TennisNumCGParams}{20}
\newcommand{\OccNumCGParams}{20}


%%%%%%%%%%%%%%%%%%%%%%
%%%%%%%%%%%%%%%%%%%%%%
%%%%%%%%%%%%%%%%%%%%%%
% Figures and tables



\newcommand{\TracesARM}{

\begin{knitrout}
\definecolor{shadecolor}{rgb}{0.969, 0.969, 0.969}\color{fgcolor}\begin{figure}[!h]

{\centering \includegraphics[width=0.98\linewidth,height=0.653\linewidth]{figure/traces_arm_graph-1} 

}

\caption[Optimization traces for the ARM models]{Optimization traces for the ARM models.  Black dots show the termination point of each method. Dots above the horizontal black line mean that DADVI found a better ELBO. Dots to the right of the vertical black line mean that DADVI terminated sooner in terms of model evaluations.}\label{fig:traces_arm_graph}
\end{figure}

\end{knitrout}
}


\newcommand{\TracesNonARM}{

\begin{knitrout}
\definecolor{shadecolor}{rgb}{0.969, 0.969, 0.969}\color{fgcolor}\begin{figure}[!h]

{\centering \includegraphics[width=0.98\linewidth,height=0.653\linewidth]{figure/traces_nonarm_graph-1} 

}

\caption[Traces for non-ARM models]{Traces for non-ARM models.  Black dots show the termination point of each method. Dots above the horizontal black line mean that DADVI found a better ELBO. Dots to the right of the vertical black line mean that DADVI terminated sooner in terms of model evaluations.}\label{fig:traces_nonarm_graph}
\end{figure}

\end{knitrout}
}




\newcommand{\RuntimeARM}{

\begin{knitrout}
\definecolor{shadecolor}{rgb}{0.969, 0.969, 0.969}\color{fgcolor}\begin{figure}[!h]

{\centering \includegraphics[width=0.98\linewidth,height=0.653\linewidth]{figure/runtimes_arm_graph-1} 

}

\caption[Runtimes and model evaluation counts for the ARM models]{Runtimes and model evaluation counts for the ARM models. Results are reported divided by the corresponding value for DADVI or LRVB.  Numbers greater than one (shown by the black line) indicate favorable performance by DADVI or LRVB.  Recall that the reported LRVB numbers include the cost of the DADVI optimization as well as the LR covariances.  Most of the ARM models are relatively low-dimensional, so the LR covariances added little to the computation.}\label{fig:runtimes_arm_graph}
\end{figure}

\end{knitrout}
}



\newcommand{\RuntimeNonARM}{

\begin{knitrout}
\definecolor{shadecolor}{rgb}{0.969, 0.969, 0.969}\color{fgcolor}\begin{figure}[!h]

{\centering \includegraphics[width=0.98\linewidth,height=0.653\linewidth]{figure/runtimes_nonarm_graph-1} 

}

\caption[Runtimes and model evaluation counts for the non-ARM models]{Runtimes and model evaluation counts for the non-ARM models. Results are reported divided by the corresponding value for DADVI or LRVB.  Numbers greater than one (shown by the black line) indicate favorable performance by DADVI or LRVB. Recall that the reported LRVB numbers include the cost of the DADVI optimization as well as the LR covariances. Missing model and method combinations are marked with an X.}\label{fig:runtimes_nonarm_graph}
\end{figure}

\end{knitrout}
}






\newcommand{\PosteriorAccuracyARM}{

\begin{knitrout}
\definecolor{shadecolor}{rgb}{0.969, 0.969, 0.969}\color{fgcolor}\begin{figure}[!h]

{\centering \includegraphics[width=0.98\linewidth,height=0.653\linewidth]{figure/posterior_arm_graph-1} 

}

\caption[Posterior accuracy measures for the ARM models]{Posterior accuracy measures for the ARM models. Each point is a single named parameter in a single model. Points above the diagonal line indicate better DADVI or LRVB performance.  Level curves of a 2D density estimator are shown to help visualize overplotting.}\label{fig:posterior_arm_graph}
\end{figure}

\end{knitrout}
}


\newcommand{\PosteriorAccuracyNonARM}{

\begin{knitrout}
\definecolor{shadecolor}{rgb}{0.969, 0.969, 0.969}\color{fgcolor}\begin{figure}[!h]

{\centering \includegraphics[width=0.98\linewidth,height=0.653\linewidth]{figure/posterior_nonarm_graph-1} 

}

\caption[Posterior accuracy measures for the non-ARM models]{Posterior accuracy measures for the non-ARM models. Each point is a single named parameter in a single model. Points above the diagonal line indicate better DADVI or LRVB performance. }\label{fig:posterior_nonarm_graph}
\end{figure}

\end{knitrout}
}


%%%%%%%%%%%%%%%%%%%%%%%%%%%%%%%%%%%%%%%%%%%%%%%%%%%%%%%%%%%%%%%%%%%%%%%%%%




\newcommand{\CoverageHistogram}{

\begin{knitrout}
\definecolor{shadecolor}{rgb}{0.969, 0.969, 0.969}\color{fgcolor}\begin{figure}[!h]

{\centering \includegraphics[width=0.98\linewidth,height=0.653\linewidth]{figure/coverage-1} 

}

\caption[Density estimates of $\Phi(\freqerr)$ for difference models]{Density estimates of $\Phi(\freqerr)$ for difference models. All the ARM models are grouped together for ease of visualization.  Each panel shows a binned estimate of the density of $\Phi(\freqerr)$ for a particular model and number of draws $\znum$. Values close to one (a uniform density) indicate good frequentist performance.  CG failed for the Occupancy and POTUS models with only 8 draws, possibly indicating poor optimization performance with so few samples.}\label{fig:coverage}
\end{figure}

\end{knitrout}
}
