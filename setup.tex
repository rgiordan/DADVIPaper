In what follows, we take data $\y$ and a finite-dimensional parameter $\theta
\in \thetadom$. We consider a user who is able to provide software
implementations of the log density of the joint distribution $\p(\y, \theta)$
and is interested in reporting means and variances from an approximation of the
exact Bayesian posterior $\p(\theta \vert \y)$.

Black-box variational inference (BBVI) refers to a spectrum of approaches for
approximating this posterior. We focus in the present paper on ADVI, a
particularly popular instance of BBVI. Variational inference forms an
approximation $\q(\theta \vert \eta)$, with variational parameters $\eta \in
\etadom$, to $\p(\theta \vert \y)$. Let $\normal{\cdot}{\mu, \Sigma}$ denote a
normal distribution with mean $\mu$ and covariance matrix $\Sigma$. The
full-rank variant of ADVI approximately minimizes the reverse KL divergence
$\textrm{KL}\left( \q(\cdot \vert \eta) || \p(\cdot \vert \y)\right)$ between
the exact posterior and an approximating family of multivariate normal
distributions:
%
\begin{align}\label{eq:advi_qdom}
%
\qdom ={}& \left\{
    \q(\theta \vert \eta):
    \q(\theta \vert \eta) =
        \normal{\theta}{\mu(\eta), \Sigma(\eta)} \right\},
%
\end{align}
%
where $\eta \mapsto (\mu(\eta), \Sigma(\eta))$ is a (locally) invertible map
between the space of variational parameters and the mean and covariance of the
normal distribution. When we optimize $\eta$ over this family, we will refer to
the resulting optimization problem as the ``full-rank ADVI optimization
problem.''

In particular, we will typically focus on the following objective function,
which is equivalent to the one above:
%
\begin{align}\label{eq:kl_objective}
%
% \etastar := \argmin_{\eta \in \etadom} \klfullobj{\eta}
% \quad\textrm{where}\quad
    % \kl{\q(\theta \vert \eta)}{\p(\theta \vert \y)}
% = \argmin_{\eta \in \etadom} \left(
\klfullobj{\eta} :=
        \expect{\q(\theta \vert \eta)}{\log \q(\theta \vert \eta)}
        -\expect{\q(\theta \vert \eta)}{\logjoint}.
        % \right),
%
\end{align}
%
The objective $\klfullobj{\eta}$ in \cref{eq:kl_objective} is equivalent to the
KL divergence $\textrm{KL}\left( \q(\cdot \vert \eta) || \p(\cdot \vert
\y)\right)$ up to $\log \p(\y)$, which does not depend on $\eta$, so that
minimizing $\klfullobj{\eta}$ also minimizes the KL divergence. The negative of
the objective, $-\klfullobj{\eta}$, is sometimes called the ``evidence lower
bound'' (ELBO) \citep{blei:2016:variational}. 

To avoid degeneracy in the objective, typically one transforms any model
parameters with restricted range before running the optimization --- and
performs the reverse transformation after. E.g., we might take the logarithm of
any strictly positive parameters so that their transformed range is the full
real line; see \citet{kucukelbir:2017:advi} for further details. Therefore, we
will henceforth assume that $\thetadom = \rdom{\thetadim}$ and $\p(\theta)$ is
supported on all $\rdom{\thetadim}$. Then in the full-rank case, $\eta$ contains
both the mean and some unconstrained representation of a $\thetadim$-dimensional
covariance matrix, so that $\eta \in \rdom{\thetadim + \thetadim (\thetadim + 1)
/ 2}$.

The mean-field variant of ADVI restricts $\Sigma(\eta)$ to be
diagonal.\footnote{Technically, in the mean-field variant of ADVI,
$\Sigma(\eta)$ may sometimes be block diagonal; see \cref{app:mean-field}.
We elide this special case in what
follows for ease of exposition. Our experiments are fully diagonal.} That is,
take the approximating family $\qdom$ to consist of independent normals with
means $\etamu \in \rdom{\thetadim}$ and log standard deviations $\etaxi \in
\rdom{\thetadim}$.
\begin{align}\label{eq:advi_qdom_mf}
%
\qdom ={}& \left\{
    \q(\theta \vert \eta):
    \q(\theta \vert \eta) =
        \prod_{d=1}^{\thetadim}
            \normal{\theta_d}{\etamu[d], \exp(\etaxi[d])^2}
\right\} \\
\etamu ={}& (\etamu[1], \ldots, \etamu[\thetadim])^\trans
\textrm{, }
\etaxi = (\etaxi[1], \ldots, \etaxi[\thetadim])^\trans
\textrm{, }
\eta = (\etamu^\trans, \etaxi^\trans)^\trans
\textrm{, }
\etadom = \mathbb{R}^{\etadim}
\textrm{, and }
\etadim = 2\thetadim.
\nonumber
%
\end{align}
%In this case, we can write the vector of variational standard deviation parameters
%as $\etasigma = \exp(\etaxi)$, where $\exp(\cdot)$ is applied
%component-wise.
For mean-field ADVI, the variational parameter $\eta^\trans = (\etamu^\trans,
\etaxi^\trans) \in \rdom{2\thetadim}$. When the variational family satisfies the
mean-field assumption, we will refer to the resulting optimization problem as
the ``mean-field ADVI optimization problem'' and its objective as the
``mean-field ADVI objective.''

Using the expression for univariate normal entropy and neglecting some
constants, the mean-field ADVI objective in \cref{eq:kl_objective} becomes
%
\begin{align}\label{eq:kl_objective_advi}
%
\klfullobj{\eta} :=
-\sum_{d=1}^{\thetadim} \etaxi[d] -
\expect{\normal{\theta}{\eta}}{\log \p(\theta, \y)}
\quad\textrm{and}\quad
\etastar := \argmin_{\eta \in \etadom} \klfullobj{\eta}.
%
\end{align}
%
We would ideally like to compute $\etastar$, but we cannot optimize
$\klfullobj{\eta}$ directly, because the term
$\expect{\normal{\theta}{\eta}}{\log \p(\theta, \y)}$ is generally intractable.
ADVI, like most current BBVI methods, employs stochastic gradient optimization
(SG) to avoid computing $\klfullobj{\eta}$.  Specifically, ADVI uses Monte Carlo
and the ``reparameterization trick'' \citep{mohamed:2020:mcgradients} as
follows. Let $\normz$ denote the $\thetadim$-dimensional standard normal
distribution. If $\z \sim \normz$, then
%if we take $\theta(\z, \eta) := \etamu + \z \odot \exp(\etaxi)$, where
%$\odot$ is the component-wise (Hadamard) product, then
%
\begin{align}\label{eq:reparameterization}
%
\expect{\normal{\theta}{\eta}}{\log \p(\theta, \y)} ={}&
    \expect{\normz}{\log \p(\etamu + \z \odot \exp(\etaxi), \y)}.
%
\end{align}
%
For compactness, we write 
%
\begin{align}\label{eq:reparameterization_theta}
    \theta(\eta, \z) := \etamu + \z \odot \exp(\etaxi),
\end{align}
%
where $\odot$ is the component-wise (Hadamard) product. 
For $\znum$ independent
draws\footnote{A subscript $\z_n$ will denote a particular member of the set
$\Z$, though for the rest of the paper, subscripts will usually denote an entry
of a vector.} $\Z := \{ \z_1, \ldots, \z_\znum \}$ from $\normz$, we can use
\cref{eq:reparameterization} to define an unbiased estimate for the mean-field
$\klfullobj{\eta}$:
%
\begin{align}\label{eq:reparameterization_kl}
%
\klobj{\eta | \Z} := 
-\sum_{d=1}^{\thetadim} \etaxi[d] -
\meann \log \p(\theta(\eta, \z_n), \y).
% \quad\textrm{and}\quad
% \etahat := \argmax_{\eta \in \etadom} \klobj{\eta | \Z}.
%
\end{align}
%
ADVI uses derivatives of $\klobj{\eta | \Z}$, with a new draw of $\Z$ at each
iteration, to estimate $\etastar$. The ADVI algorithm, which we will sometimes
refer to as ``ADVI'' in shorthand, can be found in \cref{alg:sadvi}. When we use
the ADVI algorithm for the full-rank optimization problem, we will write
``full-rank ADVI.''

%
\begin{minipage}{1.0\linewidth}
    \begin{minipage}[t]{0.49\linewidth}
        %
        \begin{algorithm}[H]
        \caption{\\ADVI (Existing method)}\label{alg:sadvi}
        \begin{algorithmic}
        \Procedure{ADVI}{}
            \State $t \gets 0$
            \State Fix $\znum$ (typically $\znum = 1$)
            \While{Not converged}
                \State $t \gets t + 1$
                \State Draw $\Z$
                \State $\Delta \gets \grad{\eta}{\klobj{\eta_{t-1} | \Z}}$
                \State $\alpha_t \gets \textrm{SetStepSize(All past states)}$
                \State $\eta_t \gets \eta_{t-1} - \alpha_t \Delta$
                \State $\textrm{AssessConvergence(All past states)}$
            \EndWhile
            \State $\tilde{\eta} \gets \eta_t$ or
                $\tilde{\eta} \gets \frac{1}{M} \sum_{t'=t- M + 1}^t \eta_{t'}$
            \State \Return $\q(\theta \vert \tilde{\eta})$
        \EndProcedure

        \Postprocessing \hspace{1em}(If possible)
            \State Assess MC error using $\eta_1, \ldots, \eta_t$
            \If{MC Error is too high}
                \State Re-run with smaller / more steps
            \EndIf
        \EndPostprocessing
        \end{algorithmic}
        \end{algorithm}
        %
    \end{minipage}
    \begin{minipage}[t]{0.49\linewidth}
        %
        \begin{algorithm}[H]
        \caption{\\DADVI: Deterministic ADVI (Our proposal)}\label{alg:dadvi}
        \begin{algorithmic}
        \Procedure{DADVI}{}
            \State $t \gets 0$
            \State Fix $\znum$ (our default is $\znum = 30$)
            \State Draw $\Z$
            \While{Not converged}
                \State $t \gets t + 1$
                \State
    $\Delta \gets \textrm{GetStep}(\klobj{\cdot | \Z}, \eta_{t-1})$
                \State $\eta_t \gets \eta_{t-1} + \Delta$
                \State $\textrm{AssessConvergence}(\klobj{\cdot | \Z}, \eta_t)$
            \EndWhile
            \State $\etaopt \gets \eta_t$
            \State \Return $\q(\theta \vert \etaopt)$
        \EndProcedure
        \Postprocessing
            \State Compute LR covariances (\cref{sec:linear_response})
            \State Assess MC error (\cref{sec:mc_error_estimation})
            \If{MC Error is too high}
                \State Re-run with more samples in $\Z$
            \EndIf
        \EndPostprocessing
        \end{algorithmic}
        \end{algorithm}
        %
    \end{minipage}
\end{minipage}

\vspace{1em}
