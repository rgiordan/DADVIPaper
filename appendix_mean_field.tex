In practice, the mean-field assumption in variational inference need not always
correspond to factorization over every single one-dimensional component of each
parameter. Rather, it often represents a factorization into individual
parameters as described in a model. For instance, consider a parameter within a
model that represents a distribution over $K$ outcomes, so that its elements are
positive and sum to one. A natural prior for such a parameter might be a
Dirichlet distribution. If this parameter exists as one parameter among multiple
parameters in our model, a mean-field assumption will typically provide a
separate factor for this parameter, but it will not further factorize across
components within the parameter. So $\Sigma(\eta)$ may, in fact, be
block-diagonal rather than purely diagonal, where each block size will
correspond to the size of a parameter. 

Researchers have explored other options between the extremes of the mean-field
and full-rank assumptions for Gaussian approximations within variational
inference; see, for instance, \citep{zhang:2022:pathfinder}.