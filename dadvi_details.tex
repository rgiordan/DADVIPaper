Our method, DADVI, will start from the same optimization objective as ADVI, but
it will use a different approximation to handle the intractable objective. As we
have seen, in ADVI, each step of the optimization draws a new random variable.
The key difference in our method, DADVI, is that the random approximation is
instead made with a single set of draws and then fixed throughout optimization.
The full DADVI algorithm appears in \cref{alg:dadvi}. In the notation of
\cref{eq:reparameterization_kl}, for a particular $\Z$, 
the value $\etahat$ returned by DADVI in \cref{alg:dadvi} is given by
%
\begin{align}\label{eq:dadvi_estimate}
    %
    \etahat := \argmin_{\eta \in \etadom} \klobj{\eta | \Z}.
    %
\end{align}
%
The $\etahat$ of \cref{eq:dadvi_estimate} is an estimate of $\etastar$ insofar
as its objective $\klobj{\eta}$ is a random approximation to the true objective
$\klfullobj{\eta}$. In general, the idea of DADVI can be applied to either the
mean-field or full-rank ADVI optimization problem (though see
\cref{sec:dadvi_full_rank} below for some potential challenges when using DADVI
with full-rank ADVI).  In what follows, analogously to how we refer to the ADVI
algorithm, we will assume that we are targeting the mean-field problem with
DADVI unless explicitly stated that we are instead targeting the full-rank
problem.


For DADVI, the reparameterization of \cref{eq:reparameterization} is essential:
it allows us to use the same set of draws $\Z$ for any value of $\eta$. With
$\Z$ fixed, $\klobj{\cdot | \Z}$ in turn remains fixed throughout optimization.
This consistency would not be possible in general without a reparameterization
like \cref{eq:reparameterization} to separate the stochasticity from the shape
of $\q(\theta \vert \eta)$.  %However, not all BBVI methods use a
%reparameterization like \cref{eq:reparameterization} to approximate the
%objective function's gradient. For example, the ``score method'' rewrites the
%objective function gradient by differentiating inside the expectation integral,
%which is then approximated using draws from $\q(\theta \vert \eta)$ at the
%current value of $\eta$
%\citep{paisley:2012:variationalstochasticsearch,ranganath:2014:bbvi,mohamed:2020:mcgradients}.
%Since the sampling distribution $\q(\theta \vert \eta)$ is different at every
%step of the optimization, it is not immediately obvious how to construct a
%single approximate objective function that remains fixed throughout optimization
%for BBVI approximations that rely on the score method.  We leave the question of
%how to apply the SAA to such problems for future work.

Note that, for a given $\Z$, all derivatives of $\klobj{\eta | \Z}$ required by
either DADVI or ADVI can be computed using automatic differentiation and a
software implementation of $\log \p(\theta, \y)$.  In this sense, both DADVI and
ADVI are black-box methods. 
In practice, another key difference between ADVI and DADVI is that ADVI
typically draws only a single random variable per iteration, whereas DADVI uses
a larger number of draws; in particular, the default number of draws for DADVI
in our experiments will be $\znum = \DADVINumDraws$.

We will see in what follows that using DADVI instead of ADVI can reap large
practical benefits.
