
\CoverageHistogram{}

We next show that frequentist standard error estimates from DADVI
provided good estimates of the sampling variability of the DADVI mean estimates, particularly
for $\znum \ge 32$.

As discussed in \cref{sec:mc_error_estimation}, the sampling variability of
DADVI estimates are straightforward to compute using standard formulas for the
sampling variability of M-estimators. For the DADVI mean estimates, we computed
the sampling standard deviation as described in
\cref{sec:mc_error_estimation,sec:lr_mc_computation}.\footnote{For the large
POTUS, Occupancy, and Tennis models, we used CG to compute frequentist coverage
for the same select quantities of interest for which we computed LR
covariances.} We denote by $\freqsd$ our estimate of
$\sqrt{\var{\normz}{\mu_\dadvi}}$ as computed using \cref{eq:mc_variance}, that
is, of the sampling standard deviation of the DADVI mean estimate under sampling
of $\Z$.  We can evaluate the accuracy of $\freqsd$ by computing $\mu_\dadvi$
with a large number of draws, which we denote as $\mu_\infty$, and checking
whether
%
\begin{align*}
%
\freqerr := \frac{\mu_\dadvi - \mu_\infty}{\freqsd}
%
\end{align*}
%
has an approximately standard normal distribution under many draws of
$\mu_\dadvi$.  We evaluated $\mu_\infty$ by taking the average of 100 runs with
$\znum = 64$ each.\footnote{The values shown in the $\znum = 64$ panel of
\cref{fig:coverage} are the same as those whose average was taken to estimate
$\mu_\infty$.  In theory, this induces some correlation between the $\freqerr$
values for $\znum = 64$.  However, the sampling variability of $\mu_\infty$ was
so small that the induced correlation is practically negligible.}

To evaluate whether $\freqerr$ has a normal distribution, we can take  $\Phi$ to
be the cumulative distribution function of the standard normal distribution, and
check whether $\Phi(\freqerr)$ has a uniform distribution.  Since the parameters
returned from a particular model are not independent under sampling from $\Z$,
the $\Phi(\freqerr)$ are not independent, and standard tests of uniformity like
the Kolmogorov-Smirnov test are not valid.
%
However, we can visually inspect the quality of the standard errors by checking
whether $\Phi(\freqerr)$ has an approximately uniform distribution, without
attempting to quantify how close it should be to uniform by chance alone. As can
be seen in \cref{fig:coverage}, for $\znum = 8$ and $\znum = 16$ the
$\Phi(\freqerr)$ values are over-dispersed to varying degrees for different
models; this behavior indicates that the sampling variance $\xi$ is under-estimated. In
contrast, the intervals provide good marginal coverage when $\znum \ge 32$,
though some over-dispersion remains in the Occupancy model.
%
