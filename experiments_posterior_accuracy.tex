\PosteriorAccuracyARM{}
\PosteriorAccuracyNonARM{}

We next see that the quality of posterior mean estimates for DADVI and the ADVI
methods are comparable. The LRVB posterior standard deviations are much more
accurate than the ADVI methods, including full-rank ADVI. 

Each method produced a posterior mean estimate for each model parameter,
$\mu_\method$, and a posterior standard deviation estimate, $\sigma_\method$.
Above, we used $\mu$ to denote the posterior expectation of the full $\theta$
vector, but here we are using it more generically to denote a posterior
expectation of some sub-vector of $\theta$, or even the posterior mean of a
transformed parameter as estimated using Monte Carlo draws from the variational
approximation in the unconstrained space. We use the NUTS estimates, $\mu_\nuts$
and $\sigma_\nuts$, as the ground truth to which we compare the various
variational methods.  In order to form a common scale for the accuracy of the
posterior means and variances, we define the relative error in the posterior
mean and standard deviation as follows:
%
\begin{align*}
%
\muerr_\method :={} \frac{\mu_\method - \mu_\nuts}{\sigma_\nuts}
\quad\textrm{and}\quad
\sderr_\method :={} \frac{\sigma_\method - \sigma_\nuts}{\sigma_\nuts}.
%
\end{align*}
%
For example, if, on a particular parameter of a particular model, we find that
$\norm{\muerr_\dadvi} < \norm{\muerr_\sadvi}$, we would say that DADVI has
provided better mean estimates of that model parameter than mean-field ADVI. For
posterior covariances we will always report $\sderr_\lrvb$ rather than
$\sderr_\dadvi$, since we expect $\sigma_\dadvi$ to suffer from the same
deficiencies as the ADVI methods due to their shared use of the
mean-field approximation.

As discussed in \cref{sec:setup}, any parameters with restricted ranges will
typically be transformed before running ADVI. In our plots, then, we include one
point each for the original and transformed versions, respectively, of each
distinctly named parameter in the PyMC model. For Occupancy, Tennis, and POTUS,
we reported posterior mean accuracy measures for all parameters, but posterior
uncertainty measures only for a small number of quantities of interest. When a
named parameter is multi-dimensional, we report the norm of the error vector
over  all dimensions in order to avoid giving too much visual weight to a small
number of high-dimensional parameters.

The posterior accuracy results for ARM and the larger models are shown
respectively in \cref{fig:posterior_arm_graph,fig:posterior_nonarm_graph}. Recall that,
of
the non-ARM models, only the Microcredit model was small enough for full-rank
ADVI.

The estimates for the posterior means are comparable across methods, with
RAABBVI performing the best on average.  However, there are parameters for which
RAABBVI's mean estimates are off by up to a hundred standard deviations while
the DADVI estimates are fairly accurate.  In contrast, when the DADVI mean
estimates are severely incorrect, the RAABBVI ones are also severely incorrect.
This pattern suggests that severe errors in the DADVI posterior means are
primarily due to the mean-field approximation, whereas severe errors in ADVI
methods can additionally occur due to problems in optimization.

The LRVB posterior standard deviation estimates are almost uniformly better than
the ADVI and RAABBVI estimates based on the mean-field approximation. This
performance is not surprising since the mean field approximation is known to
produce poor posterior standard deviation estimates.\footnote{Note that the
relative standard deviation errors for ADVI tend to cluster around 1 because
MFVB posterior standard deviations tend to be under-estimated, and so a small
posterior standard deviation estimate leads to a relative error of one.}
Interestingly, for the ARM models, even the full-rank ADVI posterior covariance
estimates are worse than the LRVB covariance estimates, which is probably due to
the difficulty of optimizing the full-rank ADVI objective.
