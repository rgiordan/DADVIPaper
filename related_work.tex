As discussed above in \cref{sec:intro}, the idea of approximating an intractable
optimization objective $F(\eta) := \expect{\normz}{f(\eta, \z)}$ by
$\hat{F}(\eta | \Z) := \meann f(\eta, \z_n)$ is well-studied in the optimization
literature as the ``sample average approximation'' (SAA) \citep[][Chapter
5]{nemirovski:2009:sgdvsfixed,royset:2013:optimalsaa,
kim:2015:guidetosaa,shapiro:2021:lectures}. A key theoretical conclusion of the
optimization literature is that, in general, SAA should perform worse than SG in
high dimensions in terms of computational cost of providing an accurate optimum.
Our theoretical results of \cref{sec:high_dim} and experimental results of
\cref{sec:experiments} suggest that these general-case analyses may be unduly
pessimistic for many BBVI problems, though we believe more work remains to be
done establishing guarantees for SAA applied to BBVI in high dimensions.

The present work and the concurrent work by \citet{burroni:2023:saabbvi}
together form the first systematic studies of the accuracy of SAA for BBVI,
though the idea of applying SAA to BBVI has occurred several times in the
literature in the context of other methodological results
\citep{giordano:2018:covariances,domke:2018:importanceweightingvi,domke:2019:divideandcouplevi,
broderick:2020:automatic,wycoff:2022:sparsebayesianlasso,giordano:2023:bnp}. The
methods and experiments of \citet{burroni:2023:saabbvi} provide a complement to
our present work. \citet{burroni:2023:saabbvi} propose and study a method for
iteratively increasing the number of draws used for the SAA approximation until
a desired accuracy is reached (see also \citet{royset:2013:optimalsaa} for a
similar approach in the optimization literature); in contrast, we keep the
number of draws fixed in our theoretical analysis and our experiments.
Additionally, the models considered by \citet{burroni:2023:saabbvi} are
relatively low-dimensional, which allow the authors to use a very large number
of draws (up to $\znum = 2^{18}$) without incurring a prohibitive computational
cost.  In contrast, almost all of our experiments in \cref{sec:experiments} use
$\znum = \DADVINumDraws$; only in our investigation of Monte Carlo error in
\cref{sec:experiments_sampling_variability} do we examine changing $\znum$, and
there we consider only $\znum$ up to 64.  Studying relatively lower-dimensional
models with a large number of draws allows \citet{burroni:2023:saabbvi} to apply
SAA with the full-rank approximation (see our discussion of the SAA with the
full-rank approximation in \cref{sec:dadvi_full_rank}).  In contrast, we
emphasize the computation of LR covariances with the SAA approximation (as in
\citet{giordano:2018:covariances}) and on the use of DADVI in higher dimensions
more generally.  One could imagine combining our approaches: for example, by
computing the size of the sampling error relative to the LR covariance, and
increasing the number of draws as necessary as recommended in
\citet{burroni:2023:saabbvi}, though we leave such a synthesis for future work.
