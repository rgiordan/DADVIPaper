The preceding sections demonstrated that, in certain cases, DADVI can work well
to estimate the optimum of the mean-field ADVI problem even in high dimensions.
By contrast, we now show that DADVI will behave pathologically for the full-rank
ADVI problem in high dimensions unless a prohibitively large number of draws are
used. The intuition we develop for full-rank ADVI also extends to other highly
expressive variational approximations such as normalizing flows.

In forming the full-rank optimization problem, ADVI parameterizes $\q(\theta |
\eta)$ using a mean $\etamu$ and a $\thetadim \times \thetadim$ matrix $\etar$
in place of $\etasigma$.  Formally, the full-rank approximation taking
$\theta(\eta, \z) = \etamu + \etar \z$ in place of the mean-field
reparameterization is given in \cref{eq:reparameterization_theta}. Letting
$|\cdot|$ denote the matrix determinant, the KL divergence becomes
%
\begin{align}\label{eq:reparameterization_fr_kl}
    %
    \klobj{\eta | \Z} := -\frac{1}{2} \log |\etar \etar^\trans| -
        \meann \log \p(\etamu + \etar \z_n, \y).
    %
\end{align}
%
The preceding display can be compared with the corresponding mean-field
objective in \cref{eq:reparameterization_kl}.
%
For the present section, we will
write $\klobj{\eta | \Z} = \klobj{(\mu, \etar) | \Z}$.  Under this
parameterization, $\cov{\q(\theta \vert \eta )}{\theta} =
\cov{\normz}{\theta(\eta, \z)} = \etar \etar^\trans$, so the matrix $\etar$ can
be taken to be any square root of the covariance matrix of $\q(\theta \vert
\eta)$.  In practice, $\etar$ is typically taken to be lower-triangular (i.e., a
Cholesky decomposition), though the particular form of the square root used will
not matter for the present discussion.

Suppose we are attempting to optimize the full-rank ADVI problem with DADVI when
$\thetadim > \znum$, so that $\theta$ has more dimensions than there are draws
$\z_n$.  Our next result shows that, in such a case, DADVI will behave
pathologically.

\begin{theorem}\label{thm:fradvi}
%
Consider a full-rank ADVI optimization problem with $\thetadim > \znum$. Then,
for any $\mu$, we have $\inf_{\etar} \klobj{(\mu, \etar) | \Z} = -\infty$, so
the DADVI estimate is undefined.
%
\begin{proof}
%
In the full-rank case, the objective function $\klobj{\eta | \Z}$ in
\cref{eq:reparameterization_fr_kl} depends on $\etar$ only through the products
$\etar \z_n$ and the entropy term, which is $\frac{1}{2} \log |\etar
\etar^\trans| = \log |\etar|$.  Since $\znum < \thetadim$, we can write $\etar =
\etar^{\Z} + \etar^{\perp}$, where $\etar^{\Z}$ is a rank-$\znum$ matrix
operating on the subspace spanned by $\Z$ and $\etar^\perp$ is a
rank-$(\thetadim - \znum)$ matrix satisfying $\etar^\perp z_n = 0$ for all
$n=1,\ldots,\znum$. Then we can rewrite the DADVI objective as
%
\begin{align}
%
\klobj{\eta | \Z}  = -\log |\etar^{\Z} + \etar^{\perp}| -
    \meann \log \p(\etamu + \etar^\Z \z_n, \y). 
    \label{eq:full_rank_dadvi_objective}
%
\end{align}
%
Since $\sup_{\etar^{\perp}} \log |\etar^{\Z} + \etar^{\perp}| = \infty$, the
result follows.\footnote{Recall that the log determinant is the sum of the logs of the
eigenvalues of $\etar$, which can be made arbitrarily large as $\etar^{\perp}$
varies freely.}
%
\end{proof}
%
\end{theorem}

What will happen, in practice, if one tries to use DADVI in the full-rank case?
Denote the maximum a posteriori (MAP) estimate as $\hat\theta :=
\argmax_{\theta} \log \p(\theta, \y)$, and note that the first term on the right
hand side of \cref{eq:full_rank_dadvi_objective} is most negative when $\etamu =
\hat\theta$ and $\etar^\Z$ is the zero matrix.  A zero $\etar^\Z$ is
impermissible because, when $\etar^\Z$ is actually the zero matrix, then
$\etar^{\Z} + \etar^{\perp}$ is singular, and $\log |\etar^{\Z} + \etar^{\perp}|
= -\infty$.\footnote{ Indeed, if $\etar^\Z = 0$, then $\q(\theta \vert \eta)$
would have zero variance in any direction spanned by $\Z$, $\p(\theta, \y)$
would not be absolutely continuous with respect to $\q(\theta \vert \eta)$, and
the reverse KL divergence would be undefined.} However, for any $\varepsilon >
0$ and $M > 0$, we can take $\etar^\Z \z_n = \varepsilon \z_n$ and $\etar^\perp
v = M v$ for any $v \perp \Z$, so that $\etar$ is full-rank. 
When $\etamu = \hat\theta$, one can always decrease both terms on the right hand
side of \cref{eq:full_rank_dadvi_objective} via the following two-step
procedure. First, decrease $\varepsilon$ by any amount and thereby decrease the
first term. Second, given that $\varepsilon$, increase $M$ by a sufficient
amount to decrease the second term as well.

The degeneracy described in \cref{thm:fradvi} can be avoided if one uses at
least as many draws as there are model parameters, i.e., if $\znum \ge
\thetadim$.  However, based on the classical optimization results discussed
above (e.g. \citet[Chapter 5]{shapiro:2021:lectures}), one might expect
full-rank ADVI to require $\znum$ to be on the order of $\thetadim^2$, since the
full-rank variational parameters have dimension of order $\thetadim^2$ due to
the inclusion of a full-rank covariance matrix.  We proved above that the
classical dimension dependence of $\znum$ on the dimension of the variational
parameters is unnecessarily pessimistic for certain mean-field ADVI objectives.
It is an interesting question for future work to ask whether the classical
dimension dependence is also pessimistic for the full-rank approximation: that
is, whether DADVI for full-rank ADVI actually requires $\znum$ to be on the
order $\thetadim$ rather than $\thetadim^2$, or somewhere in between.

Finally, we note that the failure of DADVI in the full-rank case appears to be
indicative of a general phenomenon. Any smooth function mapping the columns of
$\Z$ into $\thetadom$ must span an $\znum$-dimensional sub-manifold of
$\thetadom$.  If a variational approximation is rich enough to increase the
entropy to an arbitrary degree on the complement of this submanifold, then DADVI
will lead to a degenerate solution. In this sense, it is in fact the
inexpressivity of the mean-field variational assumption that allows DADVI to
work in high dimensions.