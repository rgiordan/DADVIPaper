\subsection{ARM models}

We selected $\ARMNumModels$ from the \texttt{Stan} example models
repository\footnote{\url{https://github.com/stan-dev/example-models/tree/master/ARM}}.
The models we used are as follows, with their parameter dimension in parentheses:
%
% https://tex.stackexchange.com/questions/422197/latex-environment-to-write-in-plain-text-mode
\newenvironment{simplechar}{%
   \catcode`\_=12
}{}

% \begin{verbatim}
%     radon_chr
% \end{verbatim}

\texttt{\ArmModels{}}.

Some models were eliminated from consideration for being duplicates of other
models, and a small number were eliminated for poor NUTS performance (low
effective sample size or poor R hat).

\subsection{Tennis}

In the tennis model, each player, $i=1,...,M$ has a rating $\theta_i$. These
ratings are drawn from a prior distribution with a shared variance:

\begin{align}
    \theta_i \stackrel{iid}{\sim} \mathcal{N}(0, \sigma^2),
\end{align}

The standard deviation $\sigma$ is given a half-Normal prior with a scale
parameter of 1. The likelihood for a match $n=1,...,N$ between player $i$ and
$j$ is given by:

\begin{align}
    y_n  \sim \text{Bernoulli}(\text{logit}^{-1}(\theta_i - \theta_j))
\end{align}

where $y_n = 1$ if player $i$ won, and $y_n = 0$ if not.




\subsection{Occupancy model}

In occupancy models, we are interested in whether site $i$ is occupied by
species $j$. We model occupation as a binary latent variable
$y_{ij}$, with probability $\Psi_{ij}$ being the probability that the species is
occupying the site. The logit of this probability is modeled as a linear
function of environmental covariates, such as rainfall and temperature:

\begin{align}
y_{ij} \sim \textrm{Bern}(\Psi_{ij}), \\
  \textrm{logit}(\Psi_{ij}) = \bm{x}^{\text{(env)}\intercal}_i \bm{\beta}^{\text{(env)}}_j + \gamma_j, \\
  \bm{\beta}^{\text{(env)}}_j \stackrel{iid}{\sim} \mathcal{N}(0, I), \\
  \gamma_j \stackrel{iid}{\sim} \mathcal{N}(0, 10^2).
\end{align}

However, $y_{ij}$ is assumed not to be observed directly. Instead, we observe
the binary outcome $s_{ijk}$, which equals one if species $j$ was observed at
site $i$ on the $k$-th visit. If the species was observed, we know that it is
present ($y_{ij} = 1$), assuming there are no false positives. If it was not, it
may have been missed, and we model the probability that it would have been
observed if it had been present, $p_{ijk}$. Mathematically speaking, these
assumptions result in the following model:

\begin{align}
  & p(s_{ijk} = 1 \mid y_{ij} = 1) = p_{ijk}, \label{eq:det-prob} \\
  & p(s_{ijk} = 1 \mid y_{ij} = 0) = 0, \label{eq:false-pos} \\
  & \textrm{logit}(p_{ijk}) = \bm{x}_{ik}^{\text{(obs)}\intercal} \bm{\beta}^{\text{(obs)}}_j \label{eq:det-prob-model},
\end{align}

where $\bm{x}_{ik}^{\text{(obs)}\intercal}$ are a set of covariates assumed to
be related to the probability of observing the species, and
$\bm{\beta}^{\text{(obs)}}_j$ are coefficients of a linear model relating these
to the logit of the probability $p_{ijk}$.

As $y_{ij}$ is not observed, it has to be marginalized out for ADVI models to be
applicable. The resulting likelihood is given by:
%
\begin{align}
  p(s \mid \theta) = \prod_{i=1}^{N} \prod_{j=1}^{J} \left[(1 - \Psi_{ij}) \prod_{k=1}^{K_i} (1 - s_{ijk}) +
  \Psi_{ij} \prod_{k=1}^{K_i} (p_{ijk})^{s_{ijk}} (1 - p_{ijk})^{1 - s_{ijk}} \right].
  \label{eq:observed-data-likelihood}
\end{align}
%
Its derivation can be found in the appendix of \cite{ingram:2022:occupancy}.
Here, $K_i$ are the number of visits to site $i$, $N$ is the total number of
sites, $J$ is the total number of species, and the rest of the variables are as
defined previously.
